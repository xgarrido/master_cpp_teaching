\documentclass[10pt,a4paper,twoside]{report}

\input{../../base/BasePackage.sty}
\input{../../base/OptionsListingC++.sty}

% Mettre des hyperliens dans le pdf
\hypersetup{pdftitle={Projet jeu d'échec},
  pdfsubject={Projet d'informatique - Magistère de Physique Fondamentale},
  pdfauthor={Xavier Garrido, LAL,
    <garrido@lal.in2p3.fr>},
  pdfkeywords={jeu, échec, C++}
}

\begin{document}
\renewcommand{\chaptername}{Projet}

\setcounter{chapter}{1}
\chapter{Jeu d'échecs}
\label{projet::jeu_echec}

Ce projet consiste à programmer un jeu d'échecs en ligne de
commande. Il n'est évidemment pas question de créer une "intelligence
artificielle" : la partie se jouera entre humains consentants.

Le programme pourra être conçu dans cet ordre :

\begin{itemize}

\item[\textbullet] un échiquier,

\item[\textbullet] une classe abstraite \lstinline$Piece$ définissant
  les méthodes de base telles que le déplacement,

\item [\textbullet] des classes filles telles que \lstinline$Tour$,
  \lstinline$Pion$~\ldots, héritant de \lstinline$Piece$ et
  surdéfinissant les méthodes préalablement déclarées,

  \item[\textbullet] les règles élémentaires telles que la prise de
    pièce, l'échec et l'échec et mat.
\end{itemize}

Accessoirement, on pourra s'intéresser :

\begin{itemize}

\item[\textbullet] au roque,
\item[\textbullet] à la prise en passant,
\item[\textbullet] à la promotion du pion.

\end{itemize}

De même, si la programmation d'une intelligence artificielle, on
pourra envisager de jouer contre l'ordinateur, cer dernier jouant
aléatoirement une pièce.

\paragraph{Remarques :} Suggestion de présentation de l'échiquier en ligne de commande (\url{http://fr.wikipedia.org/wiki/Echec})

  \begin{lstlisting}[basicstyle=\ttfamily\footnotesize]

        +----+----+----+----+----+----+----+----+
    8   | *K |    |    |    |  *R|    |    |    |
        |----|----|----|----|----|----|----|----|
    7   |    | *P |    |    |    |  Q |    |    |
        |----|----|----|----|----|----|----|----|
    6   | *P |    |    |    |    |    |    |    |
        |----|----|----|----|----|----|----|----|
    5   |    | *Q |    |    |    |    | *P |    |
        |----|----|----|----|----|----|----|----|
    4   |    |    |    |    |    |    |    |    |
        |----|----|----|----|----|----|----|----|
    3   |  P |    |    |    |    |  P |    |    |
        |----|----|----|----|----|----|----|----|
    2   |    |  P |    |    |    |    |  P |    |
        |----|----|----|----|----|----|----|----|
    1   |    |  K |    |    |    |    |    |  R |
        +----+----+----+----+----+----+----+----+
          a    b    c    d    e    f    g    h

  \end{lstlisting}

\end{document}

% Local Variables:
% mode: latex
% coding: utf-8-unix
% End:
